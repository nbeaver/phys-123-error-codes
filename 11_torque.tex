\documentclass[12pt,letterpaper]{article}
\usepackage[utf8]{inputenc}
\usepackage{amsmath}
\usepackage{siunitx}
\usepackage{gensymb}
\usepackage{enumitem}
\usepackage[compact]{titlesec}
\setlist{nosep}
\usepackage{multicol}
\usepackage[margin=0.5in]{geometry}

\begin{document}

\begin{center}
  Error codes for PHYS-123 lab \#11: ``Torque''
\end{center}

\section*{Equations.}

\begin{multicols}{2}

\section{Orthography}

\begin{enumerate}[start=10]
  \item Misspelled word.
  \item Wrong word.
  \item Missing word or phrase.
  \item Incorrect punctuation.
  \item Incorrect capitalization.
  \item Incorrect grammar.
  \item Subscripts and superscripts not formatted correctly.
\end{enumerate}

\section{General}

\begin{enumerate}[start=20]
  \item Name and lab section is missing or formatted incorrectly.
  \item Please do not use a cover sheet.
  \item This paper was not stapled.
  \item Please print your report double-sided.
  \item Your report uses multiple different fonts. Please use one font.
  \item This part of your report is hard to read.
  \item Part of your report is in the wrong order.
  \item Part of your report got cut off on the margins.
    Use larger margins when printing.
  \item Title of lab is wrong.
\end{enumerate}

\section{Introduction}

\subsection*{Equations}

\begin{enumerate}[start=30]
  \item Missing one or more equations.
  \item You need to explain the meaning of all the symbols in your equations.
  \item Do not use an asterisk ($*$) or cross ($\times$) or dot ($\bullet$)
    to represent multiplication.
    Do not use a divide sign ($\div$) for division.
  \item Equations are not on their own line.
  \item Missing numbering on equation.
  \item Equations are numbered, but formatting is wrong.
  \item Use Greek symbols instead of Latin letters,
    such as $\rho$ (rho) instead of $p$ for density.
  \item This equation was not used in your analysis.
  \item Your references do not include a source for this equation.
  \item This equation is in the wrong place. Equations go in the introduction.
\end{enumerate}

\section{Experimental methods}

\begin{enumerate}[start=40]
  \item Missing an experimental methods section.
  \item Your experimental methods are missing key information
    that is necessary to reproduce your experiment.
  \item Your procedure gives a ``how'', but not enough ``why''.
  \item Do not start the methods section with an equipment list.
    Instead, write out your experimental methods in sentences and paragraphs,
    mentioning the equipment you used as necessary.
  \item Use paragraphs for your experimental methods, not a numbered list.
  \item Say what you actually did in lab using the past tense.
    Don't just give a list of instructions to follow,
    and do not use the imperative mood.
  \item These phrases sound too much like the lab manual.
  \item Your raw data does not agree with your experimental methods.
  \item This part of your methods is confusing or unclear.
\end{enumerate}

\section{Results and Discussion}

\subsection*{Tables (general)}

\begin{enumerate}[start=50]
  \item Missing all or part of a table.
  \item This table is unnecessary.
  \item This table should be replaced by a graph.
  \item Do not split tables across multiple pages.
  \item Do not put units in every single cell of a table.
    Instead, put it units in the header of the table.
  \item Part of your table got cut off at the edges.
  \item Title or caption of table is not descriptive enough.
    If you did multiple data series,
    you need to explain what is different in each table.
  \item Your table looks suspiciously similar to someone else's table.
    Learn how to make your own tables instead of copying someone else's.
\end{enumerate}

\subsection*{Tables (numeric values)}

\begin{enumerate}[start=60]
  \item Do not list every trial on your table
    unless you need to refer to specific values in your discussion.
    Instead, use a spreadsheet to calculate the average and standard deviation
    and use those values to summarize your best estimate and uncertainty.
  \item Missing units.
  \item Wrong units (e.g. \si{N.m} instead of \si{N/m}, or using \si{J} for momentum).
  \item Did not convert units correctly,
    e.g. when converting from \si{cm} to \si{m}.
  \item Dimensionally correct units, but awkward or nonstandard form,
    e.g. \si{N.s} instead of \si{kg.m/s} for momentum
    or \si{kg/s^2} instead of \si{N/m} for spring constant.
  \item Sign error (plus or minus needs to be flipped).
  \item Some numbers are not present in raw data.
\end{enumerate}

\subsection*{Tables (uncertainty and precision)}

\begin{enumerate}[start=70]
  \item Missing uncertainty in measurements.
  \item Uncertainty is unreasonably small or large.
  \item Uncertainty calculations are incorrect.
  \item Not enough significant figures.
    Set your data acquisition software to a higher precision,
    typically at least four or five decimal points.
  \item Too many significant figures.  Round or truncate calculated values
    to the correct number of significant figures
    instead of just using whatever the software spits out.
\end{enumerate}

\subsection*{Graph}

\begin{enumerate}[start=80]
  \item Missing a graph.
  \item Graph does not plot the correct quantities.
  \item Graph plots correct quantities, but reversed the axes.
  \item Graph plots data versus itself or a transformation of itself.
    This will give an $R^2 = 1$, but it does not mean anything.
  \item A graph with trial number on any axis
    is almost always meaningless.
  \item This graph needs more data points.
  \item This graph was not necessary.
  \item This graph looks suspiciously similar to someone else's graph.
    Learn how to plot your own graphs instead of copying someone else's.
    %TODO: \item You can combine these graphs.
\end{enumerate}

\subsection*{Graph format}

\begin{enumerate}[start=90]
  \item This graph is hand-drawn instead of using computer software.
  \item This graph has no caption.
    Please include a caption to explain what the graph is plotting.
  \item Don't split captions across pages.
  \item Graph axis needs to start at zero.
  \item Missing axis numbering.
  \item Missing axis label.
  \item Wrong axis label.
  \item Axis label is missing units.
  \item Axis label has wrong units.
  \item Missing error bars.
  \item Error bars do not match uncertainty.
  \item Do not put gridlines on your graph. Just use tick marks on your axes.
  \item Do not number or label the points in your graph.
  \item Do not connect the points on your graph with a line plot.
    Use ``Points Only'', not ``Points and Lines''.
  \item The aspect ratio of your graph is distorted.
  \item This graph has compression artifacts.
    Try exporting a PNG file instead of a JPEG file.
\end{enumerate}

\subsection*{Graph fitting}

\begin{enumerate}[start=110]
  \item Missing fit line.
  \item Points are connected together instead of using a fit line.
  \item Missing equation for fit line.
  \item Missing $R^2$ value on fit line.
  \item The value for the slope or y-intercept
    has more significant figures than the data supports.
  \item Do not force the y-intercept of a fit line to zero.
  \item Please take at least three or four data points
    if you are fitting the data.
  \item This graph is not linear,
    and you were not expected to fit it.
  \item Multiple data series are present,
    but they are fitted as if they belong to the same data series.
    This means the fit values are meaningless.
\end{enumerate}

\subsection*{Discuss quantitative results.}

\begin{enumerate}[start=120]
  \item I can't tell how you calculated these values.
    If you don't know how to show your calculations with a word processor,
    you may attach handwritten work that shows one value
    and then use a spreadsheet,
    but you must show your calculations somehow.
  \item Equation is wrong.
  \item Numeric calculations are wrong.
  \item Don't repeat the questions in the lab manual.
  \item You need to interpret the meaning of the slope for the fit.
  \item You did not interpret the units of slope correctly.
  \item You did not interpret the meaning of the y-intercept of the fit.
\end{enumerate}

\subsection*{Discuss quantitative errors.}

\begin{enumerate}[start=130]
  \item You need to calculate the standard deviation
    for a series of trials. Don't forget to include units.
  \item You did not compare your calculated value to the accepted value.
  \item You did not \emph{numerically} compare your calculated value
    to the accepted value using percent error or percent difference.
  \item Plus or minus sign of percent error is incorrect.
  \item These values are significantly different when they should be close,
    but you have not adequately explained why.
  \item You did not give a specific cause for your uncertainty.
  \item Phrases like ``human error'' are too vague to be meaningful.
    Think about how you could tell someone else how to get better data
    in concrete, meaningful ways.
  \item Please distinguish random error from systematic error.
  \item Rounding error is not an acceptable reason for systematic error
    in any of the experiments for this course.
    Proper use of a spreadsheet (or even a pocket calculator)
    will result in negligably small roundoff error.
\end{enumerate}

\section{Conclusion}

\begin{enumerate}[start=140]
  \item Missing conclusion.
  \item Missing a summary of most significant \emph{numeric} results.
  \item Missing a ``lessons learned'' or
    ``what I would do differently next time'' section.
  \item Too long.
    It should be two paragraphs long at the most.
  \item conclusion introduces new information about e.g. sources of errors,
    rather than summarizing the most significant results.
  \item Too generic.
    The conclusion should be a capsule summary,
    not boilerplate that is applicable to almost any results.
    Please be more specific.
\end{enumerate}

\section{References}

\begin{enumerate}[start=150]
  \item No references at all.
    If you used an equation or compared your value to a tabulated value,
    you need a reference for it.
  \item Missing some references.
  \item Does not match APA format.
  \item You referenced a website but did not include the URL.
  \item A formal references section was not necessary for this report.
\end{enumerate}

\section*{Raw data}

\begin{enumerate}[start=160]
  \item Do note write in pencil when recording data.
  \item Do not use white-out correction fluid on your data sheet.
  \item Your data wasn't stamped and initialed by your TA.
    You won't get credit without this.
  \item Please remove the fringe of your data sheet
    if you are tearing your sheet out of a spiral notebook.
  \item You need to measure more decimal points
    so that your trials differ measurably.
  \item You probably forgot to tare your sensor.
\end{enumerate}

\section*{Extra codes}

\begin{enumerate}[start=200]
  \item In methods, you need to say that you plotted $\omega$ versus time
    and use a linear fit to calculate angular acceleration.
  \item The experimental value of moment of inertia for the platform by itself
    requires a plot of $\tau = m g r$ on the y-axis and $\alpha$ on the x-axis,
    giving a slope of moment of inertia.
  \item The solid cuboid model for the rail gives $I = \frac{m}{12} (L^2+w^2)$.
    Compare this to the experimental using percent difference.
  \item When torque is held constant and moment of inertia is varied,
    you need a graph with either:
    \begin{enumerate}
      \item $I = I_{rail} + I_{counterweights}$ on the y-axis
        and $\frac{1}{\alpha}$ on the x-axis; or
      \item $\alpha$ on the y-axis and $\frac{1}{I}$ on the x-axis.
    \end{enumerate}
    In either case the slope will be torque,
    which you can compare to your measured value $\tau = m g r$
    using percent difference.
  \item The lever arm is the spool \emph{radius},
    not the spool diameter or any other lengths.
  \item $I_{counterweight} = \frac{m L^2}{6} + m d^2$,
    where $L$ is the length of a side
    and $d$ is the distance from the center of mass of the block
    to the axis of rotation.
\end{enumerate}

\end{multicols}

\end{document}
